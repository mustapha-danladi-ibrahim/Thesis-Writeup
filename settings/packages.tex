%% packages

\usepackage{blindtext} % needed for creating dummy text passages
\usepackage{amsmath} % needed for command eqref
\usepackage{amssymb} % needed for math fonts
\usepackage{natbib}
\usepackage{multicol}
\usepackage{lscape}
\usepackage{adjustbox}
\usepackage{listings}
\usepackage{xcolor}
\usepackage{algpseudocode}
\usepackage{algorithm}
\usepackage{pythontex}
\usepackage{titlesec}
\usepackage{subcaption}
\usepackage{fancyhdr}


\usepackage[
	colorlinks=true
	,breaklinks
	%,ngerman
	]{hyperref} % needed for creating hyperlinks in the document, the option colorlinks=true gets rid of the awful boxes, breaklinks breaks lonkg links (list of figures), and ngerman sets everything for german as default hyperlinks language
\usepackage[hyphenbreaks]{breakurl} % benötigt für das Brechen von URLs in Literaturreferenzen, hyphenbreaks auch bei links, die über eine Seite gehen (mit hyphenation).
\usepackage{xcolor}
\definecolor{c1}{rgb}{0,0,0} % blue
\definecolor{c2}{rgb}{0,0.3,0.9} % light blue
\definecolor{c3}{rgb}{0,0,0} % red blue
\hypersetup{
    linkcolor={c1}, % internal links
    citecolor={c2}, % citations
    urlcolor={c3} % external links/urls
}
%\usepackage{cite} % needed for cite
\usepackage{url}
\usepackage{cleveref}
\usepackage[T1]{fontenc} 
%\usepackage{natbib}% needed for cite and abbrvnat bibliography style
\usepackage[nottoc]{tocbibind} % needed for displaying bibliography and other in the table of contents

\usepackage{graphicx} % needed for \includegraphics
\graphicspath{{figures/}{../figures/}}

\usepackage{longtable} % needed for long tables over pages
\usepackage{bigstrut} % needed for the command \bigstrut
\usepackage{enumerate} % needed for some options in enumerate
\usepackage{todonotes} % needed for todos
\usepackage{makeidx} % needed for creating an index
\makeindex
\usepackage{tocloft}

\renewcommand\cftchapafterpnum{\vskip0pt}
\renewcommand\cftsecafterpnum{\vskip0pt}